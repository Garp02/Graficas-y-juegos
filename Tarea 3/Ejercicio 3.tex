\textbf{Ejercicio 3}
\begin{enumerate}[label = (\alph*)]
	\item Demuestre que si $|E| > \binom{|V|-1}{2}$, entonces $G$ es conexa.

	\item Para cada $n > 3$ encuentre una gráfica inconexa de orden
		  $n$ con $|E| = \binom{n-1}{2}$.
\end{enumerate}

\textbf{Solución (b):}

Tomemos de referencia la cantidad de aristas de una gráfica completa.

\begin{align*}
	\binom{n}{2} &= \frac{n(n-1)}{2}
\end{align*}

Y cláramente $\binom{n}{2} > \binom{n-1}{2}$.

Por lo tanto, podemos construir gráficas inconexas de orden $n$ con	
$|E| = \binom{n-1}{2}$ eliminando $n-1$ aristas de un vértice de una 
gráfica completa $K_n$.

\begin{align*}
	\frac{n(n-1)}{2} - (n-1) &= \frac{n(n-1)}{2} - \frac{2(n-1)}{2} \\
	\\
	&= \frac{n(n-1)-2(n-1)}{2} \\
	\\
	&= \frac{(n-1)(n-2)}{2} \\
	\\
	&= \frac{(n-1)(n-2)(n-3)!}{2(n-3)!} \\
	\\
	&= \frac{(n-1)!}{2((n-1)-2)!} \\
	\\
	&= \binom{n-1}{2} \\ 
\end{align*}

Esto asegura que la gráfica sea inconexa, pues existe un vértice de grado 
cero, es decir, un vértice aislado.
 


\textbf{Demostración (a):}

Por la construcción de \textbf{(b)} se tiene que si una gráfica de n vértices 
tiene $\binom{n-1}{2}$ aristas, entonces es inconexa. 

Por lo que, si agregamos por lo menos una arista a la gráfica, entonces 
la gráfica será conexa. Y en caso de que la gráfica no tenga un vértice 
aislado, entonces, la gráfica será conexa, así que se sigue cumpliendo.

Por lo tanto si $|E| > \binom{|V|-1}{2}$, entonces $G$ es conexa.

\QED

