\textbf{Ejercicio 1.} Sea $G$ una gráfica. Demuestre que si $e \in E$, entonces 
$\mathcal{C}_G \le \mathcal{C}_{G-e} \le \mathcal{C}_G + 1$.

\textbf{Demostración:}

Sin pérdida de generalidad, supongamos que $G$ es conexa ($\mathcal{C}_G = 1$). Esto es válido porque:

Si $G$ tiene $k$ componentes, cualquier arista $e$ pertenece a una componente conexa. Al eliminar $e$, 
solo se afectará dicha componente, manteniendo inalteradas las demás.

Consideramos dos casos:

\textbf{Caso 1:} $e$ \textbf{no es un puente}
\begin{itemize}
    \item Si $e$ pertenece a un ciclo, su eliminación no desconecta la gráfica
    \item $G - e$ mantiene la conexidad: $\mathcal{C}_{G - e} = 1 = \mathcal{C}_G$
\end{itemize}

\textbf{Caso 2:} $e$ \textbf{es un puente}
\begin{itemize}
    \item Al eliminar $e$, la componente conexa que contenía $e$ se divide en dos
    \item Por tanto: $\mathcal{C}_{G - e} = 2 = \mathcal{C}_G + 1$
\end{itemize}

En ambos casos, se cumple que $\mathcal{C}_G \le \mathcal{C}_{G-e} \le \mathcal{C}_G + 1$.

\QED

\newpage