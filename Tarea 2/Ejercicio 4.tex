\textbf{Ejercicio 4.} Demuestre que para cualesquiera dos trayectorias de longitud
máxima en una gráfica conexa tiene un vértice en común.

\textbf{Demostración:}

Supongamos que $P$ y $P'$ son dos trayectorias disjuntas de 
longitud máxima en $G$. Sea $p\in P$ y $p'\in P'$, entonces, al ser $G$ conexa, existe 
un $pp'-$camino que los une. Sea $q$ el vértice en común entre las trayectorias $P$ y $P'$, 
entonces, $P$ y $P'$ pueden ser extendidas a trayectorias más largas, lo cual es una contradicción, 
pues las trayectorias son de longitud máxima. Por lo tanto, $P$ y $P'$ no son trayectorias disjuntas, 
es decir, tienen un vértice en común.  

\QED

\newpage