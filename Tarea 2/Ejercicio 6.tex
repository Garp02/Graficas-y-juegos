\textbf{Ejercicio 6.} Demuestre que si $|E|\geq |V|$, entonces G contiene un ciclo.

\textbf{Demostración por contradicción:}

Supongamos que G es una gráfica sin ciclos y que $|E|\geq |V|$

Si G no tiene ciclos y es conexo, sabemos que el número máximo de aristas que puede tener es |V| - 1, ya que G tiene que ser un árbol, y un árbol con n vértices tiene exactamente n - 1 aristas.

Si G no es conexo (tiene varios componentes conexos), cada componente conexo sigue siendo un árbol y, por lo tanto, si hay K componentes, la cantidad total de aristas en G es como máximo

$|E| \leq |V| - K$

Como K $\geq$ 1, esto implica que $|E| \leq |V| - 1$\textbf{!}

Por hipótesis teníamos que $|E|\geq |V|$, pero acabamos de deducir que en una gráfica sin ciclos se cumple $|E| \leq |V| - 1$. Esto es una contradicción, porque $|E|$ no puede ser mayor o igual que $|V|$ y, al mismo tiempo, menor que $|V|-1$.

Por lo anterior nuestra suposición inicial de que G no tiene ciclos es falsa. Por lo tanto, \textbf{$G$ debe contener al menos un ciclo}

\QED

