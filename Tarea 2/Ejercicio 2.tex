\textbf{Ejercicio 2.} Demuestre que si G tiene diámetro mayor a 3, entonces 
$\overline{G}$ tiene diámetro menor que 3.

\textbf{Demostración:}

Sea $G$ una gráfica con diámetro mayor a 3, entonces existe un par de vértices 
$u,v$ tal que la distancia entre ellos es mayor a 3. Esto implica que no existe 
un camino de longitud 2 entre $u$ y $v$ en $G$. 

Por definición, $\overline{G}$ tiene los mismos vértices que $G$, y dos vértices 
son adyacentes en $\overline{G}$ si y solo si no son adyacentes en $G$. 

Supongamos que $\overline{G}$ tiene diámetro 3 o mayor. Entonces existen dos
vértices $x,y$ en $\overline{G}$ tal que la distancia entre $x$ y $y$ es 3. 
Esto implica que $x$ y $y$ no son adyacentes en $G$. 

Si $x$ y $y$ no son adyacentes en $G$, entonces existe un vértice $w$ tal que 
$x$ y $w$ son adyacentes en $G$ y $w$ y $y$ son adyacentes en $G$. 

Por lo tanto, $w$ es adyacente a $x$ y $y$ en $\overline{G}$, lo cual implica 
que la distancia entre $x$ y $y$ en $\overline{G}$ es 1, lo cual es una 
contradicción. 

Por lo tanto, $\overline{G}$ tiene diámetro menor que 3.

\QED
