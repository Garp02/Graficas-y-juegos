\textbf{Ejercicio 5.} Caracterice las gráficas k-regulares para $k\in\{0,1,2\}$.

\textbf{Para $k = 0$:}

Sea $G$ una gráfica $0$-regular, entonces para todo $v\in V_G$, se tiene que
$deg(v)=0$. Por lo tanto, $G$ es una gráfica de $n$ vértices y $0$ aristas.

\textbf{Para $k = 1$:}

Sea $G$ una gráfica $1$-regular, entonces para todo $v\in V_G$, se tiene que 
$deg(v)=1$. Por lo tanto, $G$ es una gráfica de $n$ vértices y $n$ aristas.


\textbf{Para $k = 2$:}

Sea $G$ una gráfica $2$-regular, entonces para todo $v\in V_G$, se tiene que 
$deg(v)=2$. Por lo tanto, $G$ es una gráfica de $n$ vértices y $n$ aristas. Además, 
$G$ es una gráfica conexa, pues para todo vértice $v\in V_G$, se tiene que $deg(v)\geq 1$. 
Por lo tanto, $G$ es un ciclo.